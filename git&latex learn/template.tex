\documentclass[UTF8]{ctexart}
\CTEXsetup[format={\Large\bfseries}]{section}%默认一级标题居左
\usepackage{geometry}
\geometry{left=2.5cm,right=2.5cm,top=2.5cm,bottom=2.5cm} %页边距
\usepackage{graphicx}%图形
\usepackage{fancyhdr}%页眉页脚
\pagestyle{fancy}	%启用fancy风格设置
\lhead{}
\chead{}
\rhead{\bfseries\textsl{\today}  } % textsl 斜体
\lfoot{}
\cfoot{\thepage}
\rfoot{}
%\renewcommand{\headrulewidth}{0.6pt}    %单线页眉的设置 
\renewcommand{\footrulewidth}{0.4pt}     %单线页脚的设置 
%-----------双线页眉的设置  
\makeatletter % 进入“内部命令模式”(允许使用 @ 符号的 LaTeX 内部变量)
\def\headrule{{\if@fancyplain\let\headrulewidth\plainheadrulewidth\fi%
		\hrule\@height 1.0pt \@width\headwidth\vskip1pt%上面线为1pt粗  
		\hrule\@height 0.5pt\@width\headwidth  %下面0.5pt粗            
		\vskip-2\headrulewidth\vskip-4pt}      %两条线的距离1pt        
		  \vspace{3mm}}     %双线与下面正文之间的垂直间距 
\makeatother    % 退出“内部命令模式”
%------------双线页眉的设置            
% \usepackage{booktabs}
% \usepackage{subfigure}
\usepackage{setspace}
\usepackage{amsmath}
\usepackage{array}%需要该宏包
\usepackage{diagbox} % 加载宏包
\usepackage{multirow}
\usepackage{textcomp}
\usepackage{indentfirst}%首行缩进宏包
\usepackage{setspace}
\usepackage{amssymb}
\title{{\heiti  第一周实验:Git、LaTeX }\vspace{-2em}}
\date{}
\begin{document}
\thispagestyle{empty}  %用于设置 “当前页” 的页眉页脚风格:
\begin{figure}[tph!] %封面标题
	\centering
	\includegraphics[width=0.7\linewidth]{figure/2}
	
\end{figure}

\begin{center}% % 内容居中环境(所有内部内容均居中对齐)
	\quad \\ %插入一个小空格
	\quad \\
	\quad \\
	\quad \\
	% \quad \\
	% \quad \\
	\heiti \fontsize{30}{17} \quad \quad 第\quad 一\quad 周\quad \quad \quad 
	\vskip 0.5cm
	\songti \zihao{2} 实\quad 验\quad 报\quad 告%在此打印论文题目,二号黑体	
\end{center}
\vskip 1cm

\begin{quotation}
	\songti \fontsize{20}{20}
	\doublespacing
	\par\setlength\parindent{12em}
	\qquad
\begin{center}
		{\Large 学\hspace{0.88cm} 院:\underline{\hbox to 58mm{信息科学与工程学部\hfill}}}
		\vskip 0.3cm	
		{\Large 班\hspace{0.88cm} 号:\underline{\hbox to 58mm{计科一班\hfill}}}
		\vskip 0.3cm
		{\Large 姓\hspace{0.88cm} 名:\underline{\hbox to 58mm{顾晓宁\hfill}}}
		\vskip 0.3cm	
		{\Large 学\hspace{0.88cm} 号:\underline{\hbox to 58mm{24020007036\hfill}}}
		\vskip 0.3cm	
		{\Large 实验编号:\underline{\hbox to 58mm{第一周实验报告\hfill}}}
		\vskip 0.3cm	
		{\Large 指导教师:\underline{\hbox to 58mm{周小伟\hfill}}}
	\end{center}
	% \vskip 3cm
	% \begin{flushright}% 日期右对齐
	% 	% 2019\;年\;5\;月\;14\;日
	% 	\today
	% \end{flushright}
	
\end{quotation}
\newpage
%\thispagestyle{empty}
\tableofcontents % 自动生成目录(基于后续的 \section、\subsection 等章节命令)
\newpage
\maketitle	
\thispagestyle{fancy}	
\section{实验目的}
\textbf{练习使用Git进行版本控制,掌握LaTeX的基本使用方法}
\section{练习内容与结果}
% \begin{enumerate}
% 	\item 
% 	\item 
% 	\item 
% 	\item 
% 	\item 
% \end{enumerate}
\section{实验感悟}
% \begin{thebibliography}{9}%宽度9
% 	\addcontentsline{toc}{section}{参考文献}%将“参考文献加入目录中”
% 	\bibitem{bib:one} 蒋立平,姜萍,谭雪琴,花汉兵,数字逻辑电路与系统设计(第三版)北京:电子工业出版社,2019。 
% \end{thebibliography}
\section{个人github/gitee账号}
\section{友情链接}
\end{document}